\chapter{Exercise 5}

\section{Concepts}
\subsection{Deferred Writes (No-Force)}
\begin{itemize}
	\item When a client commits, you don't immediately write data to tdisk 
	\item you defer it until the buffer is "too full"
	\item this mimics real-world behavior where disk writes are expensive
\end{itemize}

\section{No Dirty Writes (No-Steal)}
\begin{itemize}
	\item You never write uncommited data to disk
	\item only data from committed transaction can be flushed
	\item this makes recovery simple (no undo needed)
\end{itemize}

\section{Physical Logging}
\begin{itemize}
	\item You log the exact new state ofo page (called after-image)
	\item Every operation is logged before it's commited (Write Ahead Logging)
\end{itemize}

\section{Crash Recovery (Redo Only)}
\begin{itemize}
	\item If the system crashes, you: 
	\begin{itemize}
		\item read the log
		\item Find all commited transactions
		\item Reapply their writes if needed
	\end{itemize}
\end{itemize}

\section{Why this matters}
Real-world databases: 
\begin{itemize}
	\item Delay writes to optimize performance
	\item maintain logs for safety 
	\item Use recovery algorithms (like ARIES) to bring the DB back to a consistent state after a crash	
\end{itemize}

You're building a simplified version of that: 
\begin{itemize}
	\item No transaction rollback (no UNDO)
	\item just REDO commited changes
	\item Emphasizes correctness and robustness in concurrent enrionments	
\end{itemize}