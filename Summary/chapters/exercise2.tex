\chapter{Exercise 2}

\section{Surrogate Keys}
\begin{itemize}
	\item artificial primary key that has no business meaning - it exists solely to uniquely identify records in a table
	\item \textbf{Example}: Auto-incrementing ID column (like SERIAL in PostgreSQL)
	\item they never change
	\item simpler joins (single-column integer keys are efficient)
	\item avoid exposing business data in URLs/APIs	
\end{itemize}

\section{Partitioning}
When mapping object-oriented inheritance to relational tables, there are two main strategies: 

\subsection{Horizontal Partitioning (Table per Subclass)}
\begin{itemize}
	\item each subclass gets its own table containing all attributes (both inherited and specific)
	\item no parent table exits
	\item \textbf{Pros}:
	\begin{itemize}
		\item No NULL values (houses don't need apartment fields)
		\item clean separation of concerns
	\end{itemize}
	\item \textbf{Cons}:
	\begin{itemize}
		\item Duplicates shared attributes (e.g. city, postal code in both tables)
		\item harder to query "all estates" $\rightarrow$ requires \textbf{UNION}
	\end{itemize}
\end{itemize}

\subsection{Vertical Partitioning (Table per Hierarchy)}
\begin{itemize}
	\item Single table for the entire hierarchy with: 
	\begin{itemize}
		\item a \textbf{type} column to distinguish subclasses (e.g. "House", "Apartment")
		\item all possible attributes (many NULLs for subtype-specific fields)
	\end{itemize}
	\item \textbf{Pros}:
	\begin{itemize}
		\item Simple queries (no joins needed)
		\item easy to get "all estates"
	\end{itemize}
	\item \textbf{Cons}:
	\begin{itemize}
		\item Many NULL values (wastes space)
		\item No DB-level constraints to enforce subtype rules 
	\end{itemize}
\end{itemize}
